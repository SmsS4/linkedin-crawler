
% \section{مقدمه}

% در این تمرین قصد داریم تا بک‌اند وبسایت یک هواپیمایی را که در آن فروش بلیط انجام می‌شود پیاده‌سازی کنیم. در این وب‌سایت کاربر باید بتواند مبدا، مقصد، تاریخ سفر و تعداد مسافران را مشخص کند و پیشنهادهای مرتبط با جستوجوی خود را دریافت کند و سپس از بین پیشنهادهای موجود گزینه مورد نظر خود را انتخاب و خرید خود را نهایی کند. در حین نهایی سازی پرداخت، کاربر باید به درگاه بانکی برای پرداخت هدایت شود و پس از پرداخت به وب‌سایت اصلی برگردد. همچنین برای انجام این امور نیاز است تا کاربر بتواند حساب کاربری داشته باشد و احراز هویت انجام دهد.

% در ادامه این مستند به بررسی دقیق نیازمندی‌ها و مواردی که باید پیاده‌سازی شوند می‌پردازیم. در این تمرین تلاش شده است تا تمام منابع مورد نیاز در اختیار شما قرار گیرد تا بتوانید تمرکز خود را بر روی مواد درسی حفظ کنید. لذا جداول و دیدهای\footnote{View} مورد نیاز و همچنین مجموعه دادگان جعلی در اختیار شما قرار گرفته است که در ادامه به توضیح آنها نیز پرداخته خواهد شد.

% \section{منابع مفید}

% پیش از شروع توصیه می‌شود که وب‌سایت چند مورد از شرکت‌های هواپیمایی معروف را برای آشنایی با کلیت کار بررسی کنید.

% \begin{itemize}
%     \item \href{https://www.emirates.com/ir/english/}{هواپیمایی امارات}
%     \item \href{https://www.qatarairways.com/en-us/homepage.html}{هواپیمایی قطر}
% \end{itemize}

% در این تمرین نیاز است که از پایگاه‌داده PostgreSQL استفاده کنید. لذا آشنایی اولیه با پایگاه‌داده‌های رابطه‌ای و همچنین نحوه کار با آن مورد نیاز است. توصیه می‌شود که پیش از شروع تمرین مستندات زیر را مطالعه شود.

% \begin{itemize}
%     \item \href{https://medium.com/codex/intro-to-postgresql-c8da31335c34}{مقدمه‌ای بر PostgreSQL}
%     \item \href{https://www.postgresqltutorial.com}{دوره یادگیری PostgreSQL}
% \end{itemize}

% باید در جاهایی که نیاز است، از تکنیک‌های Caching استفاده کنید. برای آشنایی با Redis می‌توانید از مستندات زیر استفاده کنید.

% \begin{itemize}
%     \item \href{https://medium.com/featurepreneur/introduction-to-the-in-memory-datastore-redis-e1ba0a3d3f20}{مقدمه‌ای بر Redis}
%     \item \href{https://redis.io/docs/getting-started/}{شروع کار با Redis}
% \end{itemize}

% در این تمرین از معماری مایکروسرویس\footnote{\lr{Microservice Architecture}} استفاده می‌کنیم که می‌توانید از طریق لینک‌های زیر مطالبی را در مورد آن مطالعه کنید.

% \begin{itemize}
%     \item \href{https://medium.com/hashmapinc/the-what-why-and-how-of-a-microservices-architecture-4179579423a9}{\lr{The What, Why, and How of a Microservices Architecture}}
%     \item \href{https://medium.com/swlh/microservices-architecture-from-a-to-z-7287da1c5d28}{\lr{Microservices Architecture From A to Z}}
% \end{itemize}

% برای احراز هویت از روش \lr{Bearer Token} استفاده می‌کنیم. در این روش در زمانی کاربر نام‌کاربری و رمز عبور خود را وارد می‌کند، یک رشته تصادفی تحت عنوان Token به کاربر برگردانده می‌شود که کاربر آن را همراه با درخواست‌هایش به سمت سرور ارسال می‌کند و سرور با تطبیق آن رشته با پایگاه داده یا رمزگشایی آن، کاربر را احراز هویت می‌کند. یک کتابخانه معروف در این حوزه JWT است. از منابع زیر می‌توانید برای آشنایی بیشتر با JWT استفاده کنید.

% \begin{itemize}
%     \item \href{https://medium.com/swlh/understand-the-concept-of-jwt-json-web-tokens-4ee18682a583}{\lr{Understanding the concept of JWT — JSON Web Tokens}}
%     \item \href{https://medium.com/@sureshdsk/how-json-web-token-jwt-authentication-works-585c4f076033}{\lr{How JSON Web Token(JWT) authentication works?}}
% \end{itemize}

% در این برنامه از پروتکل معروف HTTP برای پیاده‌سازی API استفاده می‌کنیم. اگر آشنایی کافی با این پروتکل ندارید، از منابع زیر می‌توانید برای آشنایی بیشتر استفاده کنید.

% \begin{itemize}
%     \item \href{https://medium.com/@rachid1982fsb/what-is-http-5b01392678d5}{\lr{What is HTTP?}}
%     \item \href{https://www.tutorialspoint.com/http/index.htm}{\lr{HTTP Tutorial}}
% \end{itemize}

% در نهایت در مورد مسائل امنیتی که جلوتر با آنها مواجه خواهد شد، توصیه می‌کنیم که موارد زیر را مطالعه بفرمایید.
% \begin{itemize}
%     \item \href{https://oauth.net/2/}{OAuth2}
%     \item \href{https://medium.com/@pointgoal/gin-101-enable-tls-ssl-e84090aeb43}{TLS}
% \end{itemize}

% \section{شرح نیازمندی‌ها}

% در این تمرین از معماری مایکروسرویس استفاده می‌کنیم. برنامه ما از دو مایکروسرویس Auth و \lr{Ticket} تشکیل شده است. از مزایای این معماری این است که پیاده‌سازی را می‌توان به صورت جداگانه برای سرویس‌های مختلف انجام داد و آنها را به صورت جداگانه مستقر\footnote{Deploy} کرد.

% \subsection{\lr{Auth Service}}

% سرویس احراز هویت مسئول بخش ثبت نام، ورود و منطق‌های مربوط به کاربران وبسایت است. این سرویس از آنجایی که تقریبا در هر سیستمی از بخش‌های کلیدی است، نیاز است تا سرعت بالایی داشته باشد بنابرین باید با استفاده از زبان Go پیاده سازی شود. همچنین لازم است تا از تکنیک‌های Caching نیز برای بهبود کارایی این سیستم استفاده شود. جداول پایگاه‌داده این سیستم در PostgreSQL پیاده‌سازی شده است و در مخزن گیت‌هاب ضمیمه شده در اختیار شما قرار گرفته است که می‌توانید از آن استفاده کنید. هر چند شما مختارید تا از پایگاه داده مورد علاقه خود برای این منظور استفاده کنید.

% \subsubsection{موجودیت‌ها}

% در این سیستم دو جدول اصلی داریم. مورد اول و اصلی‌تر آن جدول کاربران است که به صورت زیر نعریف می‌شود.
% هر رکورد این جدول، یک کاربر است. دفت بفرمایید که user\_id به صورت خودکار به عنوان یک عدد صحیح به هر کاربر تخصیص داده ‌می‌شود. همچنین هر کاربر باید یک ایمیل و شماره تلفن متمایز داشته باشد. جنسیت کاربر نیز باید با یک تک حرف F و M مشخص شود.

% \begin{latin}
%     \begin{lstlisting}[language=SQL, belowskip=-0.8 \baselineskip]
% CREATE TABLE IF NOT EXISTS user_account
% (
%     user_id       SERIAL PRIMARY KEY,
%     email         VARCHAR UNIQUE NOT NULL,
%     phone_number  VARCHAR UNIQUE NOT NULL,
%     gender        VARCHAR(1),
%     first_name    VARCHAR,
%     last_name     VARCHAR,
%     password_hash VARCHAR
% );\end{lstlisting}
% \end{latin}

% جدول دیگر در این پایگاه داده‌ مربوط به Token های منقضی شده است. که در ادامه در مورد هر یک از آنها توضیح خواهیم داد.

% \begin{latin}
%     \begin{lstlisting}[language=SQL, belowskip=-0.8 \baselineskip]
% CREATE TABLE IF NOT EXISTS unauthorized_token (
%     user_id    INTEGER REFERENCES user_account ON DELETE CASCADE ON UPDATE CASCADE,
%     token      VARCHAR,
%     expiration TIMESTAMP
% );\end{lstlisting}
% \end{latin}

% \subsubsection{\lr{Endpoints}}

% در این سیستم شما باید اندپوینت‌های زیر را پیاده‌سازی کنید.

% \begin{enumerate}
%     \item \lr{Sign up}: این اندپوینت باید تمام اطلاعات کاربر را دریافت کند و پس از ساخت رکورد در جداول پایگاه داده، کاربر را وارد سیستم کند.
%     \item \lr{Sign in}: این اندپوینت باید با دریافت ایمیل یا شماره تلفن و رمز عبور کاربر،‌ کاربر را وارد سیستم می‌کند. منظور از ورود به سیستم این است که یک Token به کاربر نسبت داده می‌شود.
%     \item \lr{Sign out}: این اندپوینت هیچ ورودی‌ای دریافت نمی‌کند. صرفا با فراخوانی این اندوینت کاربری که داخل سیستم است باید از سیستم‌خارج شود. در ادامه در بخش ملاحظات پیاده‌سازی در مورد سازوکار این فرآیند توضیحات تکمیلی خواهیم داد.
%     \item \lr{User Info}: این اندپوینت توسط سرویس‌های دیگر برای احراز هویت کاربران استفاده می‌شود. سیستم‌های دیگر می‌توانند توکن دریافتی را به این اندپوینت ارسال کنند و اطلاعات کاربر را در جواب دریافت کنند. توجه کنید که ممکن است توکن منقضی شده باشد و در این صورت باید پاسخ مناسب برگردانده شود.
% \end{enumerate}

% \subsubsection{ملاحظات پیاده‌سازی}

% \begin{enumerate}
%     \item دقت بفرمایید که در تمام پیاده‌سازی‌های فوق، اجرای 
%     Validation\footnote{به این معنی که صحت ورودی‌ها چک شود. به عنوان مثلا نشانی ایمیل باید فرمت خاصی داشته باشد و طول آن از مقدار خاصی بیشتر نشود.}
%     مناسب برای بررسی ورودی‌های دریافتی الزامی است.
%     \item اندپوینت‌ها باید پاسخ خود را با کد وضعیت مناسب\footnote{\lr{Status Code}} برگردانند.
%     \item شما باید رمز عبور را به صورت Hash شده در پایگاه داده بنویسید.
%     \item شما باید برای تولید Token از JWT استفاده کنید. با استفاده از JWT می‌توانید مواردی از این دست که این توکن مربوط به چه کسی است و تاریخ انقضای آن تا چه زمانی است را درون خود توکن رمزنگاری کنید و در زمانی که می‌خواهید صحت توکن را بررسی کنید، تنها کافی است که آن توکن را رمزگشایی کنید. نیازی به ثبت توکن در پایگاه داده نخواهد بود.
%     \item زمانی که یک کاربر از سیستم خارج می‌شود،‌همچنان یک توکن در دست خود دارد. چون این توکن در پایگاه داده ثبت نشده است و خود آن درون خود تاریخ انقضا خود را دارد، همچنان معتبر خواهد بود. شما باید توکن‌هایی که کابر در زمان خروج از سیستم غیر معتبر می‌کند را در جدول \lr{unathorized\_token} ثبت کنید تا دیگر ورود با آنها به سیستم ممکن نباشد.
%     \item هر بار که کاربر درخواست میزند، باید بررسی شود که آیا توکن وی در جدول توکن‌های منقضی شده در پایگاه داده قرار دارد یا خیر. برای کم شدن سربار این کار، از یک Cache استفاده میکنیم تا سربار زمانی این کار کم شود. می‌توانید برای این Caching از سیستم‌هایی مانند Redis استفاده کنید. دقت بفرمایید که منطق Caching شما باید کاملا سازگار با نیازهای سیستم باشد و هیچگاه جواب اشتباه برنگرداند. منظور از جواب اشتباه این است که مثلا زمانی که یک توکن منقضی شده آن را به عنوان توکن معتبر تشخیص دهد و یا برعکس.
%     \item رمزنگاری کردن درخواست‌ها برای حفظ محرمانگی آنها در شبکه با استفاده از TLS یا SSL امتیاز مثبت اضافی به همراه خواهد داشت.
%     \item در صورت ارتباط دیگر سرویس‌ها با این سرویس با gRPC پیاده‌سازی شود، امتیاز مثبت اضافی به همراه خواهد داشت.
%     \item در صورتی که از سیستم OAuth2 که دو توکن به کاربر اختصاص می‌دهد برای پیاده‌سازی توکن استفاده کنید. امتیاز مثبت اضافه به همراه خواهد داشت.
% \end{enumerate}

% \subsubsection{\lr{OAuth2}}

% این سرویس به صورت امتیازی می‌تواند پروتکل OAuth2 را پیاده سازی کند.   
% در این پروتکل هنگامی که کاربر نام‌کاربری و رمز‌عبور خود را می‌دهد به آن دو توکن به نام‌های AccessToken و RefreshToken داده می‌شود.\\

% \noindent{\lr{\textbf{Access Token}}}

% هر توکن مدت زمان معلومی معتبر است (مثلا ۳۰ دقیقه). کاربر برای احراز هویت‌اش را باید به وسیله AccessToken انجام بدهد. یعنی برای مثال کاربر لاگین می‌کند و AccessToken را می‌گیرد. حالا هنگامی که کاربر می‌خواهد درخواستی به سرویس‌های دیگر بزند باید این توکن را نیز همراه درخواست بفرستد و سرویس تیکت این توکن را به سرویس احرازهویت می‌دهد و سرویس احرازهویت بررسی می‌کند آیا توکن معتبر است یا خیر و در صورت نیاز می‌تواند اطلاعات کاربر را نیز بدهد.

% علت وجود AccessToken این است که کاربر برای هر درخواست خود مجبور نباشد نام‌کاربری و رمز‌عبور را همراه درخواست بفرستد. این‌طوری احتمال leak شدن اطلاعات فرد کمتر می‌شود.

% علت اینکه AccessToken تا مدت زمان مشخصی معتبر است نیز این است که اگر به هر دلیلی AccessToken دست فرد دیگری افتاد مدت زمان محدودی دسترسی به اکانت فرد داشته باشد. \\

% \noindent\bf{\lr{Refresh Token}}

% همانطور که گفته شد AccessToken تا مدت زمان محدودی معتبر است. اگر کاربر بخواهد دوباره یک AccessToken جدید بگیر باید با استفاده از RefreshToken این کار را انجام بدهد.

% دقت کنید که خود RefreshToken نیز تا زمان مشخصی اعتبار دارد ولی هربار که کاربر درخواست توکن جدید کند مدت زمان باقی مانده‌ی RefreshToken تمدید می‌شود.

% \subsection{\lr{Ticket Service}}

% این سیستم مسئولیت منطق‌های مربوط به بحث فروش بلیط را بر عهده خواهد داشت. پیاده‌سازی این سیستم را باید به وسیله NodeJS انجام دهید. همچنین مشابه سرویس قبل جداول پایگاه داده مربوط به این سیستم نیز در اختیار شما قرار گرفته است و می‌توانید از آنها استفاده کنید.

% \subsubsection{موجودیت‌ها}

% در این سیستم ما مجموعه دادگاه جعلی در نظر گرفته‌ایم تا شما درگیر وارد کردن داده به سیستم نشود. این دادگان در قالب جداول CSV هستند که شما می‌توانید در جداول SQL آنها را وارد کنید. همچنین دیدهای مورد نیاز نیز طراحی شده‌اند تا درگیر Queryهای پیچیده در SQL نشوید و بتوانید با Query های ساده اطلاعات مورد نیاز را بازیابی کنید. در واقع با اینکه ساختار جداول و دید‌های این سرویس پیچیده است، ولی شما تنها با تعداد اندکی از آنها درگیر خواهید بود.

%  دید زیر تمام فرودگاه‌ها را در بر دارد. همچنین توجه کنید که ممکن است که یک شهر چند فرودگاه داشته باشد، گزینه ALL به این معنی است که تمام فرودگاه‌های آن شهر مورد قبول است. دقت بفرمایید که هر فرودگاه یک کد سه حرفی تحت عنوان کد یاتا دارد که یکتا است. به عنوان مثال کد فرودگاه بین‌المللی جان اف. کندی نیویورک JFK و کد فرودگاه بین‌المللی مهرآباد THR است.
% \begin{latin}
%     \begin{lstlisting}[language=SQL, belowskip=-0.8 \baselineskip]
% CREATE VIEW origin_destination AS
% SELECT country_name as county,
%        city_name    as city,
%        airport_name as airport,
%        iata_code    as iata
% FROM airport
% UNION
% SELECT country_name   as county,
%        city_name      as city,
%        'All airports' as airport,
%        'ALL'          as iata
% FROM city;
% \end{lstlisting}
% \end{latin}

% جدول زیر مربوط به خرید‌های کاربر است. در این جدول همچنین باید اطلاعات تراکنش بانکی منجر به خرید نیز ثبت شود. در ادامه در مورد تراکنش بانکی صحبت خواهیم کرد.

% \begin{latin}
%     \begin{lstlisting}[language=SQL, belowskip=-0.8 \baselineskip]
% CREATE TABLE IF NOT EXISTS purchase
% (
%     corresponding_user_id INTEGER,
%     title                 VARCHAR,
%     first_name            VARCHAR,
%     last_name             VARCHAR,
%     flight_serial         INTEGER REFERENCES flight ON DELETE RESTRICT ON UPDATE RESTRICT,
%     offer_price           INTEGER,
%     offer_class           VARCHAR
%     transaction_id        INTEGER,
%     transaction_result    INTEGER
% );
% \end{lstlisting}
% \end{latin}

% دید زیر مربوط به تمام پرواز‌های موجود است. نیازی به فهم این دید نیست، آشنایی با ستون‌های آن برای شما کفایت می‌کند. به سادگی با دستورات ساده SELECT می‌توانید کوئری‌های مورد نیاز سیستم را از این دید دریافت کنید.


% \begin{latin}
%     \begin{lstlisting}[language=SQL, belowskip=-0.8 \baselineskip]
% CREATE VIEW available_offers AS
% SELECT flight.flight_id                                        AS flight_id,
%        flight.origin                                           AS origin,
%        flight.destination                                      AS destination,
%        flight.departure_utc::TIMESTAMP WITH TIME ZONE AT TIME ZONE
%        (SELECT timezone_name
%         FROM airport_timezone
%         WHERE airport_timezone.iata_code = flight.origin)      AS departure_local_time,
%        (flight.departure_utc + flight.duration)::TIMESTAMP WITH TIME ZONE AT TIME ZONE
%        (SELECT timezone_name
%         FROM airport_timezone
%         WHERE airport_timezone.iata_code = flight.destination) AS arrival_local_time,
%        flight.duration                                         AS duration,
%        flight.y_price                                          AS y_price,
%        flight.j_price                                          AS j_price,
%        flight.f_price                                          AS f_price,
%        ((SELECT y_class_capacity FROM aircraft_view WHERE aircraft_view.registration = flight.aircraft) -
%         (SELECT COUNT(*)
%          FROM purchase
%          WHERE purchase.flight_serial = flight.flight_serial
%            AND offer_class = 'Y'))                             AS y_class_free_capacity,
%        ((SELECT j_class_capacity FROM aircraft_view WHERE aircraft_view.registration = flight.aircraft) -
%         (SELECT COUNT(*)
%          FROM purchase
%          WHERE purchase.flight_serial = flight.flight_serial
%            AND offer_class = 'J'))
%                                                                AS j_class_free_capacity,
%        ((SELECT f_class_capacity FROM aircraft_view WHERE aircraft_view.registration = flight.aircraft) -
%         (SELECT COUNT(*)
%          FROM purchase
%          WHERE purchase.flight_serial = flight.flight_serial
%            AND offer_class = 'F'))                             AS f_class_free_capacity,
%        aircraft_view.aircraft_type                             AS equipment
% FROM flight
%          JOIN aircraft_view on aircraft_view.registration = flight.aircraft;
% \end{lstlisting}
% \end{latin}

% \subsubsection{اندپوینت‌ها}

% در این سیستم طراحی ساختار اندپوینت‌ها بر عهده خودتان است. شما باید این اندپوینت‌ها را به صورتی پیاده‌سازی کنید که پاسخ‌گو نیاز‌های سیستم باشد. الزامی است که در پیاده‌سازی اندپوینت‌ها از Method و StatusCode مناسب استفاده کنید. همچنین اعتبارسنجی داده‌ها برای بررسی صحت آنها نیز الزامی خواهد بود.
% تعداد اندپوینت‌های مورد نیاز برای پیاده‌سازی کم است ولی ساختار آنها با توجه به صلاح دید خودتان می‌تواند متفاوت باشد.

% \section{درگاه بانکی}

% در این تمرین شما باید اتصال به یک درگاه بانکی فرضی را پیاده‌سازی کنید. یک درگاه بانکی ساده توسط دستیاران آموزشی پیاده‌سازی شده است که شما می‌توانید از طریق مخزن ضمیمه شده به آن دسترسی داشته باشید. کار با بانک جعلی پیاده‌سازی شده بسیار ساده است. دستورالعمل راه‌اندازی بانک در README مخزن مربوطه نوشته شده است.

% اگر در مرورگر خود مسیر \lr{\transaction} را وارد نمایید. با یک UI ساده از این بانک مواجه خواهید شد که می‌توانید از طریق آن تمام تراکنش‌های ثبت شده در سیستم را مشاهده کنید و یا یک تراکنش جدید در سیستم بسازید. همچنین وضعیت تراکنش‌های قبلی را نیز از این طریق می‌توانید مشاهده کنید.

% این سیستم بسیار ساده است. یک Endpoint در این سیستم وجود دارد که شما باید تراکنش را به وسیله آن در بانک تعریف کنید.

% \begin{latin}
%     \noindent POST /transaction
% \end{latin}

% در تعریف تراکنش نیاز است که شما موارد زیر را در قالب FormData در بدنه درخواست به سیستم ارسال کنید.

% \begin{center}
% \begin{tabular}{|c|c|r|}
% \hline
% \multicolumn{1}{|c|}{\textbf{کلید}} & \multicolumn{1}{c|}{\textbf{عنوان}} & \multicolumn{1}{c|}{\textbf{توضیح}}                          \\ \hline
% amount                              & مبلغ تراکنش                         &                                                              \\ \hline
% receipt\_id                        & شناسه پذیرنده                       & یک عدد دلخواه را به عنوان شناسه وبسایت خودتان در نظر بگیرید. \\ \hline
% callback                            & محل بازگشت                          & پس از اتمام تراکنش به این مکان redirect خواهید شد.           \\ \hline
% \end{tabular}
% \end{center}

% در پاسخ به این درخواست یک شناسه به شما برگردانده خواهد شد. شما در وبسایت خود باید کاربر متقاضی پرداخت را به مسیر

% \begin{latin}
%     \noindent GET /payment/<transaction\_id>
% \end{latin}

% هدایت کنید. کاربر با یک چنین صفحه‌ای مواجه خواهد شد.

% \begin{center}
%     \includegraphics[width=\textwidth]{pictures/bank.png}
% \end{center}

% پس از انجام تراکنش کاربر به مسیر

% \begin{latin}
%     \noindent GET /callback/<result>
% \end{latin}

% هدایت خواهد شد. در متغیر result نتیجه تراکنش نوشته شده است. این نتیجه می‌تواند مقادیر 1 الی ۵ را به خود بگیرد. هر یک از این مقادیر متناظر با یک حالت تراکنش است.

% \begin{enumerate}
%     \item \lr{Success}
%     \item \lr{Input Mismatch}
%     \item \lr{Expire}
%     \item \lr{No Credit}
%     \item \lr{Cancel}
% \end{enumerate}

% شما پس از هدایت شدن کاربر به وبسایت‌، باید پیغام متناسب با هر یک از نتایج به وی نمایش دهید.

% \section{پرسش و پاسخ}

% در حین حل تمرین اگر به ابهامی برخورد کردید، می‌توانید در گروه درس آن را با دستیاران آموزشی مطرح کنید. دستیاران آموزشی پاسخگوی سوالات شما خواهند بود.

% \section{ارتباط با تمرین قبلی}

% شما باید تمام Endpoint ها را به نحوی به تمرین قبلی وصل کنید تا یک وبسایت کامل بر روی سیستم لوکال خودتان داشته باشید.

% \section{یک توصیه مهم}

% ابتدا سعی کنید Backend را پیاده‌سازی و تست کنید و پس از آن عملیات وصل کردن به Front را انجام دهید. برای تست کردن API های پیاده‌سازی شده می‌توانید از Postman استفاده کنید.

% \section{نحوه تحویل}

% جزئیات تمرین‌های این درس معمولا باعث می‌شوند تا تمرین‌ها بیش از تخمین اولیه شما زمان لازم داشته باشند. توصیه می‌کنیم تا از موکول کردن تمرین به روز‌های آخر پرهیز کنید.

% یک مخزن گیت‌هاب در سازمان درس بسازید و تمام کدهای خود را در آن بارگذاری کنید. میزان مشارکت هر یک از اعضای تیم با توجه به این مخزن و commit های آن سنجیده خواهد شد.

% در نهایت در یک ویدئو کنفرانس شما تمرین خود را ارائه می‌دهید و نمره‌دهی نهایی انجام می‌شود.

% \section{مخزن‌های مربوطه}

% \begin{itemize}
%     \item مخزن بانک \href{https://github.com/Web-TechTeam-Sharif-1401/Bank.git}{\lr{GitHub}}
%     \item مخزن پایگاه داده و دادگان جعلی \href{https://github.com/Web-TechTeam-Sharif-1401/TicketDatabase.git}{\lr{GitHub}}
% \end{itemize}

% \begin{flushleft}
% سلامت باشید
% \end{flushleft}
